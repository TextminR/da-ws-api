Die Umsetzung der API kann in unterschiedlichsten Arten und Weisen erfolgen. Wichtig ist hierbei die Auswahl der richtigen Programmiersprache und des passenden Frameworks. Im folgenden werden zwei Ansätze analysiert. Eines davon ist die Umsetzung mittels Java und dem Spring-Framework. Die andere Option ist die Umsetzung mittels Python und dem FastAPI-Framework.
\\\\
Warum Java bzw. Python? 
\\
Dafür gibt es mehrere Gründe. Mike Koder erwähnt in seiner Masterarbeit: "[...] the ability to automatically infer OpenAPI documentation were detected to reduce manual repetitive work." \cite{koder2021increasing}. Dies zeigt, dass die Möglichkeit, eine automatische OpenAPI (früher bekannt als Swagger) Dokumentation zu generieren, den Arbeitsaufwand reduziert. Zusätzlich wird in der erwähnten Masterarbeit bestätigt, dass FastAPI und Spring unter den Frameworks mit den meisten Features sind und dass die Programmiersprache keinen großen Einfluss auf die Produktivität hat, sondern dass das gewählte Framework wichtiger ist. Für dieses Projekt kommen daher nur die beiden genannten Frameworks, Spring und FastAPI, infrage, da nur Erfahrung in Java und Python besteht. 


\subsection{Java \& Spring}



\subsection{Python \& FastAPI}

\subsection{Datenbanl}
https://ieeexplore.ieee.org/abstract/document/7433710